\documentclass{td}

\usepackage{src/sty/config}

\codeUE{XLG4IU020}
\intituleUE{Programmation Concurrente en Multi-Threads}
\cursus{M1 ALMA \& Smart Computing}

\author{Matthieu \textsc{Perrin}}

\logo{src/img/logoUN.png}
\institution{Nantes Université}

\typeTP[1]
\title{Introduction à la concurrence}

\hypersetup{
  pdftitle  = {TP 1 - Introduction à la concurrence},
  pdfauthor = {Matthieu Perrin},
  pdfsubject  = {TP de M1 du cours Programmation Concurrente en Multi-Threads},
  pdfkeywords = {concurrence, Java, modèle de mémoire, moniteur, parallélisme, sûreté, thread, verrou, vivacité}
}

\begin{document}

\maketitle

Téléchargez l'archive contenant le code à exécuter sur Madoc.
Pour chaque programme du package \linebreak\texttt{concurrence}, vous devez en comprendre le code,
prédire la sortie attendue, et enfin confronter vos prédictions
à l'exécution du programme.

%%%%%%%%%%%%%%%%%%%%%%%%%%%%%%%

\begin{exercice}[Asynchronie]

  Étudiez le programme \texttt{concurrence.HelloWorld}.

  \begin{question} 
  \item Prédisez-en la sortie.
  \item Exécutez le programme plusieurs fois. Comment expliquez-vous les résultats ?
  \item Utilisez le débogueur pour obtenir les sorties décrites en commentaires.
  \item Est-il possible d’obtenir le point d'exclamation avant les autres mots ?
  \end{question}
  
\end{exercice}

%%%%%%%%%%%%%%%%%%%%%%%%%%%%%%%

\begin{exercice}[Section critique]

  Étudiez le code du programme \texttt{concurrence.SharedCounter}.

  \begin{question} 
  \item Prédisez-en la sortie
  \item Exécutez le programme en faisant varier le nombre d'itérations :
    10, 100, 1000, 10000, 100000. Que constatez-vous ?
  \item Lisez la page \texttt{man javap}, puis exécutez
    \texttt{javap -c SharedCounter.class}. Expliquez les résultats
    obtenus lors de l'exécution du programme.
  \item Ajoutez le mot-clé \lstinline{synchronized} dans les en-tête des méthodes statiques \lstinline{increment} et \lstinline{decrement} puis relancez l'exécution.
  \item Proposez une définition du mot-clé Java \lstinline{synchronized}.
  \end{question}

\end{exercice}

%%%%%%%%%%%%%%%%%%%%%%%%%%%%%%%

\begin{exercice}[Vivacité]

  Étudiez le code du programme \texttt{concurrence.Friend}.
  
  \begin{question} 
  \item Exécutez le programme. Que se passe-t-il ?
  \item Utilisez le débogueur pour obtenir les sorties décrites en commentaires.
  \end{question}

\end{exercice}

%%%%%%%%%%%%%%%%%%%%%%%%%%%%%%%

\begin{exercice}[Moniteurs]

  Étudiez le code du programme \texttt{concurrence.Semaphore}.

  \begin{question} 
  \item Exécutez le programme en faisant varier la valeur de l'argument du constructeur de \lstinline{Semaphore} : 6, puis 4, puis 2. Que constatez-vous ?
  \item Proposez une définition de sémaphore.
  \item Proposez une définition de \lstinline{wait} et \lstinline{notify}.
  \end{question}

\end{exercice}

%%%%%%%%%%%%%%%%%%%%%%%%%%%%%%%

\begin{exercice}[Réentrance]

  Étudiez le code du programme \texttt{concurrence.Reentrance} (seulement la classe \texttt{Reentrance}).

  \begin{question} 
  \item Exécutez le programme. Que remarquez-vous ?
  \item Lors de la déclaration de \lstinline{lock}, remplacez \lstinline{ReentrantLock} par \lstinline{NonReentrantLock}. Que constatez-vous ?
  \item Proposez une définition de la réentrance.
  \end{question}

\end{exercice}

%%%%%%%%%%%%%%%%%%%%%%%%%%%%%%%

\begin{exercice}[Modèles de mémoire]

  Étudiez le code du programme \texttt{concurrence.Volatile}.

  \begin{question}
  \item Prédisez-en la sortie.
  \item Exécutez le programme.
  \item Ajoutez le mot-clé \texttt{volatile} devant
    la déclaration de la variable \lstinline{value} et ré-exécutez.
  \item Que fait le mot-clé \texttt{volatile} ?
  \end{question}

\end{exercice}

\end{document}

% SPDX-License-Identifier: CC-BY-SA-4.0
% Session: Java Memory Model
% Exercise: Identify data races

\begingroup

\begin{exercice}[Identification de \emph{data races}]
  \label{exo:memory/dataraces}

  On examine le programme Java suivant. Les trois variables partagées sont initialisées à \lstinline{x = 0} et \lstinline{flag = true}.
  On suppose qu'aucun autre thread n'intervient et que les réveils intempestifs possibles de \lstinline{wait()} sont correctement
  gérés par la boucle \lstinline{while}. On raisonne sous l'hypothèse de \emph{cohérence séquentielle} pour les questions 1–3.

  \begin{lstlisting}
    final class Program {
      static final Object o = new Object();
      static int x = 0;
      static boolean flag = true;

      public static void main(String[] args) throws Exception {
        Thread t = new Thread(() -> {
          x = 1;                 
          synchronized (o) {     
            flag = false;        
            o.notify();          
          }                      
          x = 2;                 
        });

        synchronized (o) {       
          t.start();             
          while (flag) {         
            o.wait();            
          }
          int a = x;             
        }                        
        int b = x;               
      }
    }
  \end{lstlisting}

  \begin{enumerate}
  \item Donnez la machine à états du programme.
  \item Déduisez-en toutes les exécutions séquentiellement cohérentes. Quelles sont les valeurs possibles de $a$ et $b$ ?
  \item Identifiez toutes les \emph{data races} présentes dans le programme.
  \item Donnez deux corrections \emph{minimales} et \emph{distinctes} qui éliminent toutes les \emph{data races} :
    \begin{itemize}
    \item en ne modifiant que des qualificatifs (ex. \lstinline{volatile}) dans les déclarations ;
    \item en ne modifiant que la synchronisation (ex. blocs \lstinline{synchronized}).
    \end{itemize}
  \item Pour chacune des deux corrections, quelles sont alors les valeurs possibles de $a$ et $b$ ?
  \end{enumerate}

\end{exercice}

\endgroup
\endinput

% SPDX-License-Identifier: CC-BY-SA-4.0
% Session: Blocking synchronization
% Exercise: The bank accounts

\begingroup

\begin{exercice}[Les comptes en banque]
  \label{exo:blocking/bank}
  
  Une banque vous demande d'implémenter son nouveau système d'information.
  Chaque compte de ses clients est représenté par un numéro de compte entier et possède un
  certain solde entier en centimes d'euros.
  Au centre de son système réparti, les comptes de ses clients sont gérés
  par une classe \lstinline{Banque} possédant les deux méthodes suivantes.

  \begin{itemize}
  \item La méthode \lstinline{int solde(int compte)} retourne le solde du compte passé en paramètre.
  \item La méthode \lstinline{boolean transfere(int montant, int de, int a)} vérifie que les fonds sont disponibles sur le compte \lstinline{de}
    (sinon, elle s'arrête en retournant \lstinline{false}), puis elle soustrait \lstinline{montant} du compte \lstinline{de} et l'ajoute
    au compte \lstinline{a} de manière atomique, puis retourne \lstinline{true}.
  \end{itemize}
  
  On supposera que l'ensemble des comptes est représenté par un tableau de taille fixe
  d'entiers positifs ou nuls. Lors de l'initialisation, on dotera chaque compte d'un montant fixe.

  On souhaite une implémentation de \lstinline{Banque} satisfaisant la contrainte suivante :
  
  \og Si deux transferts ne concernent pas les mêmes comptes, on ne veut pas empêcher le parallélisme entre les deux appels à \lstinline{transfere}.\fg

  \begin{question}
  \item Comment appelle-t-on la stratégie à mettre en place pour satisfaire la contrainte ?
  \item Proposez une implémentation de \lstinline{Banque} satisfaisant la contrainte.
  \end{question}

\end{exercice}

\endgroup
\endinput

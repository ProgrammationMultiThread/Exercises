% SPDX-License-Identifier: CC-BY-SA-4.0
% Session: Blocking synchronization
% Exercise: Pie making

\begingroup

\begin{exercice}[Confection de tartes]
  \label{exo:blocking/pipeline}

  On veut modéliser le fonctionnement d'un atelier de pâtisserie.
  Pour préparer une tarte aux fraises, les pâtissiers doivent
  effectuer plusieurs étapes modélisées par des appels de fonctions comme ceci :

  \begin{itemize}
  \item \lstinline{Fond preparerFond()} : prépare et étale un fond de tarte (20 minutes);
  \item \lstinline{Pate cuire(Fond fond)} : cuit un fond de tarte (10 minutes);
  \item \lstinline{Creme preparerCreme()} : prépare une crème pâtissière (10 minutes);
  \item \lstinline{Tarte monter(Pate pate, Creme creme)} : étale la crème sur la pâte, et dispose les fruits (15 minutes).
  \end{itemize}

  \begin{question}
  \item Proposez un programme Java parallélisant la création de tartes.
    Chaque thread représentera un pâtissier spécialisé sur une, et une seule, tâche.
    Vous embaucherez autant de pâtissier que nécessaire pour vous
    assurer qu'aucun d'entre eux ne reste inactif. 
  \end{question}

\end{exercice}

\endgroup
\endinput

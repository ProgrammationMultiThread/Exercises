% SPDX-License-Identifier: CC-BY-SA-4.0
% Session: Blocking synchronization
% Exercise: The philosophers' dinner

\begingroup

\begin{exercice}[Le dîner des philosophes%
  \footnote{Inspiration : E. W. Dijkstra, \emph{Hierarchical ordering of sequential processes}. Acta informatica (1971)}]
  \label{exo:blocking/philosophers}

  Des philosophes sont assis autour d'une table ronde. Devant
  chaque philosophe se trouve un bol de riz. Entre chaque paire de bols voisins
  se trouve une baguette. Ces philosophes occupent la
  majorité de leur temps à réfléchir, mais entre deux périodes
  d'intense réflexion, un philosophe a besoin de se nourrir. Pour
  manger, il doit commencer par prendre sa baguette de gauche et sa
  baguette de droite. Il peut alors manger un peu de riz. Une fois rassasié,
  il repose les deux baguettes. Faisons l'hypothèse que chaque
  philosophe mange pendant un temps fini et que le bol contient
  tellement de riz qu'il en restera toujours.
  
  \begin{question}
  \item En utilisant un théorème du cours, proposez une solution deadlock-free. 
  \item Cette solution est-elle starvation-free ?
  \end{question}

\end{exercice}

\endgroup
\endinput

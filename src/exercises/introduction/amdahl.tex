% SPDX-License-Identifier: CC-BY-SA-4.0
% Session: Form parallelism to concurrency
% Exercise: Amdahl's law

\begingroup

\begin{exercice}[La loi d'Amdahl]
  \label{exo:introduction/amdahl}

  Une proportion $p=70\,\%$ d'un programme séquentiel peut être parallélisée. On dispose d'une machine avec $n=10$~c\oe urs.

  \begin{question} 
  \item Quelle est l'accélération (speedup) attendue ?
  \item On trouve dans le commerce une machine avec $n=20$~c\oe urs. Quelle est l'accélération attendue ?
  \item Quelle est l'accélération maximale théorique lorsque $n \to \infty$ ?
  \end{question}

  Un ingénieur pense qu'il existe un moyen d'optimiser la partie séquentielle pour la rendre deux fois plus rapide.

  \begin{question} 
  \item Quelle proportion du programme est parallélisable dans la version otimisée ? 
  \item Quelle accélération obtient-on alors, dans la parallélisation avec $n = 10$~c\oe urs ?
  \item À coût égal, vaut-il mieux choisir d'acheter la machine avec $n=20$~c\oe urs ou payer l'ingénieur pour optimiser le code ? 
  \end{question}
  
\end{exercice}

\endgroup
\endinput

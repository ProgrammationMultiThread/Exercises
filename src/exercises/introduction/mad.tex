% SPDX-License-Identifier: CC-BY-SA-4.0
% Session: Form parallelism to concurrency
% Exercise: The mad scientist

\begingroup

\begin{exercice}[Le savant fou%
  \footnote{Inspiration : M. Herlihy \& N. Shavit. \emph{The Art of Multiprocessor Programming}. Morgan Kaufmann (2008)}]
  \label{exo:introduction/mad_scientist}

  En visite pédagogique dans un laboratoire, vous et le reste de votre groupe êtes 
  capturés par un savant fou. Il fait l'annonce suivante :

  \og
  J'ai mis en place une pièce qui contient une lumière et un interrupteur
  qui la commande. Actuellement, la lumière est éteinte.
  Vous pouvez discuter ce soir d'une stratégie commune, mais à partir de demain
  vous serez isolés et ne pourrez plus communiquer.
  Chaque jour, je tirerai au hasard de manière uniforme l'un d'entre vous, qui passera la journée dans la pièce.
  À tout moment durant sa visite, il pourra déclarer~: \emph{Nous avons tous visité la pièce}.
  S'il a raison, vous êtes libres. Sinon, je vous reboote les cerveaux !
  \fg

  Hypothèses implicites :
  \begin{itemize}
    \item La lumière est éteinte au départ.
    \item L'interrupteur est binaire (allumé/éteint).
    \item La lumière est invisible depuis l'extérieur de la pièce,
      et il n'y a aucun autre moyen de communication entre les étudiants.
  \end{itemize}
  
  \begin{question}
    \item Énoncez précisément ce que doit garantir une stratégie réussie.
    \item Proposez une stratégie et justifiez brièvement en quoi elle satisfait l'exigence précédente.
    \item Reprenez la question précédente, sans l'hypothèse sur l'état initial de la lumière.
    \item Reprenez les deux premières questions, sans l'hypothèse que le tirage est uniforme. À la place, on suppose
      que si personne ne déclare que tout le monde a visité la pièce, chaque étudiant est sélectionné une infinité de fois
      (pas nécessairement à intervalles bornés). 
  \end{question}

\end{exercice}

\endgroup
\endinput

% SPDX-License-Identifier: CC-BY-SA-4.0
% Session: Form parallelism to concurrency
% Exercise: Asynchrony

\begingroup

\begin{exercice}[Asynchronie]
  \label{exo:introduction/asynchronous}

  On se donne le programme multi-threads suivant : 

  \begin{lstlisting}
    Thread t = new Thread(() -> {
      System.out.print("a");
      new Thread(() -> { System.out.print("b"); }).start();
      System.out.print("c");
    });

    System.out.print("d");
    t.start();
    System.out.print("e");
    t.join();
    System.out.print("f");
  \end{lstlisting}

  \begin{question}
  \item Décrivez la relation ``happens-before'' du programme.
  \item Décrivez toutes les sorties possibles pour le programme, en sachant qu'il est séquentiellement cohérent.
  \item Décrivez la machine à états du programme.
  \end{question}

\end{exercice}

\endgroup
\endinput

% SPDX-License-Identifier: CC-BY-SA-4.0
% Session: Form parallelism to concurrency
% Exercise: Parallelization of matrix multiplication

\begingroup

\begin{exercice}[Parallélisation]
  \label{exo:introduction/matrix}

  Le but de cet exercice est d'écrire une fonction \lstinline{double[][] product(double[][] M1, double[][] M2)}
  qui calcule le produit de deux matrices carrées $M_1$ et $M_2$ de taille $n\times n$.
  On rappelle la définition de la multiplication des matrices. Si $M_3 = M_1 \times M_2$, alors pour tout
  $i,j \in \{0,\dots,n-1\}$,
  $$
    M_3[i][j] = \sum_{k=0}^{n-1} M_1[i][k] \times M_2[k][j].
  $$

  \begin{question}
    \item Implémentez la version séquentielle de \lstinline{product}.
    \item Proposez une parallélisation où chaque cellule $M_3[i][j]$ est calculée par un thread distinct (jusqu'à $n^2$ threads). 
    \item Proposez une version qui crée au plus $T$ threads ($T$ est une variable globale), chacun calculant un lot de cellules.
      Précisez votre stratégie de répartition.
    \item Analysez la complexité en temps des versions et justifiez l'intérêt
          de limiter le nombre de threads à une borne indépendante de $n$.
  \end{question}

\end{exercice}

\endgroup
\endinput

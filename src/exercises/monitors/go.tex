% SPDX-License-Identifier: CC-BY-SA-4.0
% Session: Synchronization through monitors
% Exercise: Go channels

\begingroup

\begin{exercice}[Le mot-clé \texttt{select} en go]
  \label{exo:monitors/go}

  En Go, un canal (\emph{channel}) est une construction du langage, proche d'une \texttt{BlockingQueue} en Java, 
  qui permet de transférer des informations entre goroutines et qui peut bloquer jusqu'à ce qu'un élément soit disponible 
  (lecture) ou qu'une place se libère (écriture). L'instruction \texttt{select} permet d'attendre plusieurs opérations 
  sur des canaux et de débloquer dès que l'une d'elles est prête. Par exemple, le programme suivant affiche le premier 
  message reçu, soit depuis \texttt{channel1}, soit depuis \texttt{channel2}.
 
  \begin{lstlisting}
    select {
      case msg1 := <-channel1: fmt.Println(msg1)
      case msg2 := <-channel2: fmt.Println(msg2)
    }
  \end{lstlisting}

  \begin{question}
  \item Il n'existe pas d'équivalent direct en Java. Concevez une abstraction qui permette d'attendre
    sur plusieurs files bloquantes (type \texttt{BlockingQueue})
    et de reprendre dès que l'une d'elles est prête.
  \end{question}

\end{exercice}

\endgroup
\endinput

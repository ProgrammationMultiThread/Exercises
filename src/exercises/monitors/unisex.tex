% SPDX-License-Identifier: CC-BY-SA-4.0
% Session: Synchronization through monitors
% Exercise: Unisex toilets

\begingroup

\begin{exercice}[Les toilettes unisexes%
  \footnote{Inspiration : A. B. Downey. \emph{The Little Book of Semaphores}. Green Tea Press (2008)}]
  \label{exo:monitors/unisex}
    
  Les toilettes unisexes peuvent être utilisées par les hommes et par les femmes, sous les contraintes suivantes :
  \begin{enumerate}
  \item il ne doit jamais y avoir en même temps des hommes et des femmes aux toilettes ;
  \item il ne doit jamais y avoir plus de trois personnes simultanément aux toilettes.
  \end{enumerate}
  
  Un programme modélisant le problème est disponible sous Madoc. Le comportement des personnes qui
  accèdent aux toilettes est déjà modélisé. Les toilettes doivent implémenter l'interface
  \lstinline{Bathroom}, qui contient une méthode \lstinline{void enter(boolean isMale)} et une méthode \lstinline{void leave(boolean isMale)}. 
  
  \begin{question}
  \item Proposez une solution deadlock-free.
  \item Modifiez votre solution pour la rendre starvation-free.
  \end{question}

  \emph{Indice :} des humains bien élevés organiseraient probablement une file d'attente.  

\end{exercice}

\endgroup
\endinput
